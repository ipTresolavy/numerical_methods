\documentclass{article}
\usepackage[utf8]{inputenc}
\usepackage{amsmath}
\usepackage[dvipdfmx]{graphicx} 
\usepackage{bmpsize}
\usepackage{bbm}



\title{MAP3122 - Métodos Numéricos e Aplicações\\ Professor Roma\\ }
\author{TAREFA \#03 --- SEMANA 03\\ Suba até 2023.02.06, 23:55h.}
\date{}

\begin{document}
\maketitle

%\noindent Professor Alexandre Roma\\
%roma ``at'' ime.usp.br\\
%SALA 288-A, telefones 3091-6144 ou 3091-6136 (SecMAP)\\
%Atendimento 3as e 5as feiras - ligue ou passe um e-mail antes.

    \vspace{-3mm}
    
    \mbox{}\hfill  --- VERSÃO 2023.01.22

\vspace{3mm}

Por favor, leia atentamente. Os objetivos principais das tarefas são o aprendizado contínuo e o amadurecimento técnico.
Embora todas as questões  possam ser resolvidas de forma individual, esta tarefa foi preparada para ser resolvida em dupla podendo até mesmo ser discutida coletivamente. Valem as seguintes orientações para a TAREFA \#03:

\begin{enumerate}
 
 \item Todos os alunos devem entregar a solução de todas as questões, independentemente se elas forem resolvidas individualmente, em dupla ou coletivamente. Coloque sempre o nome todos os autores e cite referências bibliográficas se parte do material não for de autoria da dupla. 
 \item Nunca digite as soluções de questões teóricas. Manuscreva-as, ambos os da dupla, e carregue-as no  formato pdf. IMPORTANTE: tenha sempre o original com você para uso em aula (como material de consulta).
 \item Questões computacionais, além de um arquivo pdf contendo um relato técnico sucinto (exemplos na página da disciplina), devem ser acom\-panha\-das pelo programa usado. Qualquer linguagem de programação pode ser usada. Insira trechos de código no texto só se isso for imprescindível à explicação. Dê preferência ao uso de algoritmos e deixe tais inserções para apêndices específicos os quais devem ser citados no texto (Google {\it algorithm environment}\, \LaTeX).
 \item Salvo menção em contrário, todo o método numérico usado deve ser programado pela dupla. Funções matemáticas e.g. seno, cosseno, exponencial, \dots, podem ser diretamente usadas, naturalmente.
 \item Ajude sim, sempre que possível, um colega com dificuldade mas  não preju\-dique o aprendizado dele fornecendo soluções prontas.
 \item Em caso de dúvida, use o Fórum do Estudante. Por favor, não envie suas dúvidas diretamente ao professor.
\end{enumerate}

\vspace{2mm}
\noindent {\bf ATENÇÃO:}\, algumas dentre as questões abaixo podem valer nota e/ou serem incluídas em outras atividades durante o quadrimetre. A nota máxima desta tarefa é 10.0 (dez).

\mbox{}
\vspace{5mm}



\section{Problema unidimensional (``escalar'')}

Use o Método de Euler Implícito em conjunto com o Método das Aproximações Sucessivas para resolver o problema de valor inicial
\[\frac{d}{dt}{y}(t)=e^{t-2y(t)},\quad t\in [0,1],\quad y(0)=\frac{\ln(2)}{2}.\]

\section{Problema bidimensional}
Use o Método de Euler Implícito ou o Método do Trapézio, 
\[ y_0\doteq y(t_0),\quad t_{k+1}=t_0+(k+1)\Delta t,\quad y_{k+1}=y_k+\frac{\Delta t}{2}\,\Big[f(t_k,y_k)+f(t_{k+1},y_{k+1})\Big], 
\] 
para $k=0,1,\ldots , n-1$, onde $\Delta t=(t_f-t_0)/n$,
em conjunto com o Método de Newton para resolver uma equação do tipo presa-predador. Se Trapézio, 0.5 ponto de bônus.

\section{O que é esperado?}
É esperado um relatório sucinto contendo título, autores com sua afiliação e forma de contato (e-mail profissional), introdução/objetivos, ``modelagem mate\-mática'', metodologia numérica e resultados (com a depuração e a aplicação), finalizando com uma seção de encerramento (tipicamente denominada ``conclusão'' -  no sentido de encerrar e não de deduzir). O relatório em si, sua estrutura e conteúdo vale 5.0 pontos. 

A seção de resultados valem outros 5.0 pontos. Ele deve conter ao menos duas subseções: uma de verificação por solução manufaturada e outra com aplicações aos problemas de interesse. Tabelas de convergência numérica para verificar a correta implementação e para observar o comportamento do método em aplicações para as quais não se tem solução exata conhecida são imprescindíveis. Leia as subseções 2.3.1 a 2.3.3 do CAP.2 do manuscrito na página da disciplina. No caso dos problemas com solução desconhecida, estime o erro de discretização global usando três aproximações em malhas de integração progressivamente mais finas $r\Delta t$, $\Delta t$ e $\Delta t/r$, onde $r$ é um natural maior que 2.

\vspace{5mm}


Adapte apropriadamente os programas da TAREFA \#01. Se assim, bônus de 0.5 ponto. Seu programa não deveria ser difícil de ser rodado por um outro colega assim como reobter os resultados, a partir do relato, deveria ser uma tarefa relativamente simples.

\vspace{3mm}\noindent
{\bf IMPORTANTE: claro, como na TAREFA \#01, é preciso discutir e detalhar. Tais discussões serão feitas sob demanda dos alunos. Os resultados devem ser reportados de forma organizada e de maneira que possam ser reproduzidos por qualquer colega da turma. Há exemplos de como reportar na página da disciplina (``material suplementar'') que podem servir de ponto de partida caso queira.}

\section{COMENTÁRIOS FINAIS}
    Acho importante dar um caráter ``instrumental'' à disciplina. O uso do \LaTeX\quad (e.g. via overleaf.com) e a ``boa conduta'' no traçado de gráficos têm este propósito. Boas apresentações valorizam (e muito) a produção intelectual. Trabalho coletivo saudável melhora a famosa ``dinâmica de grupo''. 
    
\vspace{5mm}
Dúvidas? Recorra ao Fórum do Estudante que é mediado pelo professor e por monitores. Deixe de lado ``dona Maria do zap''.
\end{document}


